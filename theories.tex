%------------------------------------------------------------
\begin{frame}[c]{Propositional theories}

\begin{itemize}
\item Formulas are formed from
  \begin{itemize}
  \item atoms in $\mathcal{A}$
  \item $\bot$
  \end{itemize}
using
  \begin{itemize}
  \item conjunction ($\wedge$)
  \item disjunction ($\vee$)
  \item implication ($\rightarrow$)
  \end{itemize}
\item <2-> Notation
  \begin{eqnarray*}
    \top   &=& (\bot\rightarrow\bot)
    \\
    {\neg \phi} &=& (\phi   \rightarrow\bot)
  \pause
  \end{eqnarray*}
\item <3-> A \alert{propositional theory} is a finite set of formulas
\end{itemize}

\end{frame}
%------------------------------------------------------------
\begin{frame}{Reduct}
  \begin{itemize}
  \item<1->
    The satisfaction relation $X\models \phi$ between a set~$X$ of atoms and\\
    a (set of) formula(s) $\phi$ is defined as in propositional logic
  \item<2->
    The \alert{reduct}, $\reduct{\phi}{X}$, of a formula $\phi$ relative to a set $X$ of atoms
    \pause[3]%
    is defined recursively as follows:
    \begin{itemize}
    \item <3-> % $\reduct{\phi}{X}=\bot$ \quad\qquad\qquad
      $\reduct{\phi}{X}=\rlap{$\bot$}\phantom{(\reduct{\psi}{X} \circ \reduct{\varphi}{X})}$
      \quad
      if  $X\not\models \phi$
    \item <4-> % \rlap{$\reduct{\phi}{X}=\phi$} \quad\qquad\qquad
      $\reduct{\phi}{X}=\rlap{$\phi$}\phantom{(\reduct{\psi}{X} \circ \reduct{\varphi}{X})}$
      \quad
      if  $\phi\in X$
    \item <5-> $\reduct{\phi}{X}=(\reduct{\psi}{X} \circ \reduct{\varphi}{X})$
      \quad
      if  $X\models \phi$
      and  $\phi=(\psi\circ\varphi)$ for $\circ\in\{\wedge,\vee,\rightarrow\}$
      \medskip
    \item <6->
      If $\phi=\;{\neg \psi}=(\psi\rightarrow\bot)$,\\
      then $\reduct{\phi}{X}=(\bot\rightarrow\bot)=\top$, if $X\not\models \psi$,
      and $\reduct{\phi}{X}=\bot$, otherwise
    \end{itemize}
    \medskip
  \item <7->
    The \alert{reduct}, $\reduct{\Phi}{X}$, of a propositional theory $\Phi$ relative to
    a set $X$ of atoms is defined as
    \
    \(
    \reduct{\Phi}{X}
    =
    \{\reduct{\phi}{X} \mid \phi\in \Phi\}
    \)
  \end{itemize}
\end{frame}
%------------------------------------------------------------
\begin{frame}{Stable models}
  \begin{itemize}
  \item<1->
    A set~$X$ of atoms satisfies a propositional theory~$\Phi$, written $X\models\Phi$,
    \\
    if $X\models \phi$ for each $\phi\in\Phi$
  \item<2->
    The set of all $\subseteq$-minimal sets of atoms satisfying a propositional
    theory~$\Phi$ is denoted by $\min_\subseteq(\Phi)$
    \medskip
  \item<3->
    A set $X$ of atoms is a \alert{stable model} of a propositional theory~$\Phi$,
    \\
    if $X\in\min_\subseteq(\reduct{\Phi}{X})$
    \medskip
  \item<4->
    If $X$ is a stable model of~$\Phi$, then
    \begin{itemize}
    \item $X\models\Phi$ ~and
    \item $\min_\subseteq(\reduct{\Phi}{X})=\{X\}$
    \end{itemize}
  \item<5-> \structure{Note}
    In general, this does not imply $X\in\min_\subseteq(\Phi)$!
  \end{itemize}
\end{frame}
%------------------------------------------------------------
\begin{frame}{Two examples}
  \begin{itemize}
  \item
    \(
    \Phi_1=\{p\vee(p\rightarrow (q\wedge r))\}
    \)
    \begin{itemize}
    \item <2-> For $X=\{p,q,r\}$, we get\\
      \(
      \reduct{\Phi_1}{\{p,q,r\}}=\{p\vee(p\rightarrow (q\wedge r))\}
      \)
      \onslide<3->
      and
      \(
      \min_\subseteq(\reduct{\Phi_1}{\{p,q,r\}})=\{\emptyset\}
      \)
      \KO
    \item <4-> For $X=\emptyset$, we get\\
      \(
      \reduct{\Phi_1}{\emptyset}=\{\bot\vee(\bot\rightarrow \bot)\}
      \)
      \onslide<5->
      and
      \(
      \min_\subseteq(\reduct{\Phi_1}{\emptyset})=\{\emptyset\}
      \)
      \OK
    \end{itemize}
  \item[]
  \item <6->
    \(
    \Phi_2=\{p\vee({\neg p}\rightarrow (q\wedge r))\}
    \)
    \begin{itemize}
    \item <7-> For $X=\emptyset$, we get\\
      \(
      \reduct{\Phi_2}{\emptyset}=\{\bot\}
      \)
      \onslide<8->
      and
      \(
      \min_\subseteq(\reduct{\Phi_2}{\emptyset})=\emptyset
      \)
      \KO
    \item <9-> For $X=\{p\}$, we get\\
      \(
      \reduct{\Phi_2}{\{p\}}=\{p\vee(\bot\rightarrow \bot)\}
      \)
      \onslide<10->
      and
      \(
      \min_\subseteq(\reduct{\Phi_2}{\{p\}})=\{\emptyset\}
      \)
      \KO
    \item <11-> For $X=\{q,r\}$, we get\\
      \(
      \reduct{\Phi_2}{\{q,r\}}=\{\bot\vee(\top\rightarrow (q\wedge r))\}
      \)
      \onslide<12->
      and
      \(
      \min_\subseteq(\reduct{\Phi_2}{\{q,r\}})=\{\{q,r\}\}
      \)
      \OK
    \end{itemize}
  \end{itemize}
\end{frame}
%------------------------------------------------------------
\begin{frame}{Relationship to logic programs}
\bigskip
\begin{itemize}
\item<1->
  The translation, $\tau[(\phi\leftarrow \psi)]$, of a rule~$(\phi\leftarrow \psi)$
  is defined as follows:%
  \pause[2]
  \begin{itemize}
  \item $\tau[(\phi\leftarrow \psi)]=(\tau[\psi]\rightarrow\tau[\phi])$
  \item $\tau[\bot]=\bot$
  \item $\tau[\top]=\top$
  \item $\tau[\phi]=\phi$ \qquad if $\phi$ is an atom
  \item $\tau[{\neg \phi}]=\;{\neg \tau[\phi]}$
  \item $\tau[(\phi,\psi)]=(\tau[\phi]\wedge\tau[\psi])$
  \item $\tau[(\phi;\psi)]=(\tau[\phi]\vee\tau[\psi])$
  \end{itemize}
\smallskip
\item <3-> The translation of a logic program~$P$ is $\tau[P]=\{\tau[r]\mid r\in P\}$
\medskip
\item <4-> Given a logic program~$P$ and a set~$X$ of atoms,

  $X$ is a stable model of~$P$
  iff
  $X$ is a stable model of~$\tau[P]$
\end{itemize}

\end{frame}
%------------------------------------------------------------
\begin{frame}[c]{Logic programs as propositional theories}
  \begin{itemize}
  \item<1->
    The normal logic program
    \(
    P
    =
    \{
    p \leftarrow {\neg q}, \
    q \leftarrow {\neg p}
    \}
    \)\\
    corresponds to
    \(
    \tau[P]
    =
    \{
    {\neg q}\rightarrow p, \
    {\neg p}\rightarrow q
    \}
    \)
    \begin{itemize}
    \item<2-> stable models: $\{p\}$ and $\{q\}$
    \end{itemize}
    \medskip
  \item<3->
    The disjunctive logic program
    \(
    P
    =
    \{
    p\;;q \leftarrow
    \}
    \)\\
    corresponds to
    \(
    \tau[P]
    =
    \{
    \top \rightarrow p\vee q
    \}
    \)
    \begin{itemize}
    \item<4-> stable models: $\{p\}$ and $\{q\}$
    \end{itemize}
    \medskip
  \item<5->
    The nested logic program
    \(
    P
    =
    \{
    p \leftarrow {\neg {\neg p}}
    \}
    \)\\
    corresponds to
    \(
    \tau[P]
    =
    \{
    {\neg {\neg p}}\rightarrow p
    \}
    \)
    \begin{itemize}
    \item<6-> stable models: $\emptyset$ and $\{p\}$
    \end{itemize}
  \end{itemize}
\end{frame}
% ----------------------------------------------------------------------
%
%%% Local Variables:
%%% mode: latex
%%% TeX-master: "../../main"
%%% End:
