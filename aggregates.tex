% ----------------------------------------------------------------------
\begin{frame}{Motivation}
  \begin{itemize}
  \item Aggregates provide a general way to obtain a single value from a
    collection of input values
  \item Popular aggregate (functions)
    \begin{itemize}
    \item average
    \item count
    \item maximum
    \item minimum
    \item sum
    \end{itemize}
  \item Cardinality and weight constraints rely on count and sum aggregates
  \end{itemize}
\end{frame}
%------------------------------------------------------------
\begin{frame}{Syntax}
  \begin{itemize}
  \item
    An \alert{aggregate} has the form:
    \[
    \alpha~\{ w_1:a_1,\dots,w_m:a_m,w_{m+1}:{\neg a_{m+1}},\dots,w_n:{\neg a_n} \} \prec k
    \]
    where for $1 \leq i \leq n$
    \begin{itemize}
    \item $\alpha$ stands for a function mapping multisets over $\mathbb{Z}$ to $\mathbb{Z} \cup \{+\infty, -\infty\}$
    \item $\prec$ stands for a relation between $\mathbb{Z}\cup\{+\infty, -\infty\}$ and $\mathbb{Z}$
    \item $k\in\mathbb{Z}$
    \item $a_i$ are atoms  and
    \item $w_i$ are integers
    \end{itemize}
  \item \structure{Example}
    \(
    \mathit{sum}~\{ 30:hd(a),\dots,50:hd(m) \} \leq 300
    \)
  \end{itemize}
\end{frame}
%------------------------------------------------------------
\begin{frame}{Semantics}
  \begin{itemize}
  \item
    A (positive) aggregate \ $\alpha~\{ w_1:a_1, \dots, w_n:a_n \} \prec k$
    \par
    can be represented by the formula:
    \[
    \bigwedge_{I \subseteq \{1,\dots,n\}, \alpha \{ w_i \mid i \in I \} \not \prec k}
    \left(
      \bigwedge_{i \in I} a_i \rightarrow \bigvee_{i \in \overline{I}} a_i
    \right)
    \]
    where $\overline{I} = \{1,\dots,n\}\setminus I$ and $\not\prec$ is the
    complement of $\prec$

  \item Then,
    $\alpha~\{ w_1:a_1, \dots, w_n:a_n \} \prec k$ is true in $X$
    iff\\
    the above formula is true in $X$
  \end{itemize}
\end{frame}
% ------------------------------------------------------------
\begin{frame}{Example}
  \begin{itemize}
  \item Consider $\mathit{sum}\{ 1:p,1:q\}\neq 1$

    That is, $a_1=p$,  $a_2=q$ and  $w_1=1$,  $w_2=1$

  \item Calculemus!
    \[
    \begin{array}{c|c|c|c}
      I & \{ w_i \mid i \in I \} & \sum\{ w_i \mid i \in I \}   &  \sum\{ w_i \mid i \in I \} = 1\\
      \hline
      \emptyset&\{    \}&0&\mathit{false}\\
      \{1\}    &\{ 1  \}&1&\mathit{true}\\
      \{2\}    &\{   1\}&1&\mathit{true}\\
      \{1,2\}  &\{ 1,1\}&2&\mathit{false}
    \end{array}
    \]\pause
  \item We get
    \(
    (p\to q)\wedge (q\to p)
    \)
    \pause
  \item Analogously, we obtain
    \(
    (p\vee q)\wedge \neg (p\wedge q)
    \)
    for
    $\mathit{sum}\{ 1:p,1:q\}=1$
  \end{itemize}
\end{frame}
% ------------------------------------------------------------
\begin{frame}{Monotonicity}
  \smallskip
  \begin{itemize}
  \item \structure{Monotone aggregates} \
    \begin{itemize}
    \item For instance,
      \begin{itemize}
      \item \pbody{r}
      \item $\mathit{sum}\{1:p,1:q\}>1$ \ \onslide<2-> amounts to $p\wedge q$\onslide<1->
      \end{itemize}
    \item We get a simpler characterization:
      \(
      \bigwedge_{I \subseteq \{1,\dots,n\}, \alpha \{ w_i \mid i \in I \} \not \prec k}
      \bigvee_{i \in \overline{I}} a_i
      \)
    \end{itemize}
    \smallskip
  \item \structure{Anti-monotone aggregates} \
    \begin{itemize}
    \item For instance,
      \begin{itemize}
      \item \nbody{r}
      \item $\mathit{sum}\{1:p,1:q\}<1$ \ \onslide<2-> amounts to $\neg p\wedge\neg q$\onslide<1->
      \end{itemize}
    \item  We get a simpler characterization:
      \(
      \bigwedge_{I \subseteq \{1,\dots,n\}, \alpha \{ w_i \mid i \in I \} \not \prec k}
      \neg\bigwedge_{i \in I} a_i
      \)
    \end{itemize}
    \smallskip
  \item \structure{Non-monotone aggregates} \
    \begin{itemize}
    \item For instance, $\mathit{sum}\{1:p,1:q\}\neq 1$
      is non-monotone.
    \end{itemize}
  \end{itemize}
\end{frame}
% ----------------------------------------------------------------------
%
%%% Local Variables:
%%% mode: latex
%%% TeX-master: "../../main"
%%% End:
