% ----------------------------------------------------------------------
\begin{frame}[c]{Motivation}
  \begin{itemize}
  \item Classical versus default negation
    \bigskip
    \begin{itemize}\itemsep 1ex
    \item<1-> \structure{Symbol}  $-$ and $\neg$
    \item<2-> \structure{Idea}
      \begin{itemize}
      \item $\rlap{${-a}$}\phantom{{\neg a}}\ \approx\ {-a}   \in X$
      \item ${\neg a}\ \approx   \quad a\notin X$
      \end{itemize}
    \item<3-> \structure{Example}
      \begin{itemize}
      \item $\mathit{cross}\leftarrow{-\mathit{train}}$
      \item $\mathit{cross}\leftarrow{\neg \mathit{train}}$
      \end{itemize}
    \end{itemize}
  \end{itemize}
\end{frame}
% ----------------------------------------------------------------------
\begin{frame}{Classical negation}
  \begin{itemize}
  \item<1-3> We consider logic programs in negation normal form
    \begin{itemize}
    \item That is, classical negation is applied to atoms only
    \end{itemize}
  \item<2-> Given an alphabet $\mathcal{A}$ of atoms,
    let $\overline{\mathcal{A}}=\{{-a}\mid a\in\mathcal{A}\}$
    such that $\mathcal{A}\cap\overline{\mathcal{A}}=\emptyset$
  \item<3-> Given a program $P$ over $\mathcal{A}$, classical negation is encoded by adding
    \[
    P^-
    =
    \{a \leftarrow b,{-b} \mid a\in(\mathcal{A}\cup\overline{\mathcal{A}}), b\in \mathcal{A}\}
    \]
    \smallskip
  \item<4->
    A set~$X$ of atoms is a \alert{stable model} of a program~$P$
    over $\mathcal{A}\cup\overline{\mathcal{A}}$,
    \\
    if
    $X$ is a stable model of $P\cup P^-$
  \end{itemize}
\end{frame}
% ----------------------------------------------------------------------
\begin{frame}{An example}
  \begin{itemize}
  \item <1-> The program
    \[
    P
    \ =\
    \{ a \leftarrow {\neg b}, \ b \leftarrow {\neg a}\}
    \cup
    \{c\leftarrow b,\ {-c}\leftarrow b\}
    \]
  \item<2-> [] induces
    \[
    P^-
    =
    \left\{
      \begin{array}{rclp{2pt}rclp{2pt}rcl}
          a &\leftarrow& a,{-a} &&  a &\leftarrow& b,{-b} &&  a &\leftarrow& c,{-c}\\
        {-a}&\leftarrow& a,{-a} &&{-a}&\leftarrow& b,{-b} &&{-a}&\leftarrow& c,{-c}\\
          b &\leftarrow& a,{-a} &&  b &\leftarrow& b,{-b} &&  b &\leftarrow& c,{-c}\\
        {-b}&\leftarrow& a,{-a} &&{-b}&\leftarrow& b,{-b} &&{-b}&\leftarrow& c,{-c}\\
          c &\leftarrow& a,{-a} &&  c &\leftarrow& b,{-b} &&  c &\leftarrow& c,{-c}\\
        {-c}&\leftarrow& a,{-a} &&{-c}&\leftarrow& b,{-b} &&{-c}&\leftarrow& c,{-c}\\
      \end{array}
    \right\}
    \]
  \item<3-> The stable models of $P$ are given by the ones of $P\cup P^-$, viz $\{a\}$
  \end{itemize}
\end{frame}
%------------------------------------------------------------
\begin{frame}{Properties}
  \bigskip
  \begin{itemize}
  \item The only inconsistent stable ``model'' is~$X=\mathcal{A}\cup\overline{\mathcal{A}}$
  \item <2->[] Strictly speaking,
    an inconsistemt set like $\mathcal{A}\cup\overline{\mathcal{A}}$ is not a model
    \medskip
  \item <3-> For a logic program~$P$ over $\mathcal{A}\cup\overline{\mathcal{A}}$,
    exactly one of the following two cases applies:
    \begin{enumerate}
    \item All stable models of~$P$ are consistent or
    \item $X=\mathcal{A}\cup\overline{\mathcal{A}}$ is the only stable model of~$P$
    \end{enumerate}
  \end{itemize}
\end{frame}
%------------------------------------------------------------
\begin{frame}[c]{Train spotting}

\begin{itemize}\itemsep 5pt
\item<1-2,13>
  \(
  P_1=\{\mathit{cross}\leftarrow{\neg \mathit{train}}\}
  \)
  \begin{itemize}
  \item<2,13> stable model: $\{\mathit{cross}\}$
  \end{itemize}
\item<1,3-4,13>
  \(
  P_2=\{\mathit{cross}\leftarrow{-\mathit{train}}\}
  \)
  \begin{itemize}
  \item<4,13> stable model: $\emptyset$
  \end{itemize}
\item<1,5-6,13>
  \(
  P_3=\{\mathit{cross}\leftarrow{-\mathit{train}}, \
          {-\mathit{train}}\leftarrow\}
  \)
  \begin{itemize}
  \item<6,13> stable model: $\{\mathit{cross},{-\mathit{train}}\}$
  \end{itemize}
\item<1,7-8,13>
  \(
  P_4=\{\mathit{cross}\leftarrow{-\mathit{train}}, \
          {-\mathit{train}}\leftarrow, \
          {-\mathit{cross}}\leftarrow\}
  \)
  \begin{itemize}
  \item<8,13> stable model: $\{\mathit{cross},{-\mathit{cross}},\mathit{train},{-\mathit{train}}\}$
  \end{itemize}
\item<1,9-10,13>
  \(
  P_5=\{\mathit{cross}\leftarrow{-\mathit{train}}, \
          {-\mathit{train}}\leftarrow{\neg \mathit{train}}\}
  \)
  \begin{itemize}
  \item<10,13> stable model: $\{\mathit{cross},{-\mathit{train}}\}$
  \end{itemize}
\item<1,11-12,13>
  \(
  P_6=\{\mathit{cross}\leftarrow{-\mathit{train}}, \
          {-\mathit{train}}\leftarrow{\neg \mathit{train}}, \
          {-\mathit{cross}}\leftarrow\}
  \)
  \begin{itemize}
  \item<12,13> no stable model
  \end{itemize}
\end{itemize}

\end{frame}
% ----------------------------------------------------------------------
\begin{frame}{Default negation in rule heads}

  \begin{itemize}
  \item<1-3> We consider logic programs with default negation in rule heads
  \item<2-> Given an alphabet $\mathcal{A}$ of atoms,
    let
    \(
    \widetilde{\mathcal{A}}=\{\widetilde{a}\mid a\in\mathcal{A}\}
    \)
    such that $\mathcal{A}\cap\widetilde{\mathcal{A}}=\emptyset$
  \item<3-> Given a program $P$ over $\mathcal{A}$, consider the program
    \begin{eqnarray*}
      \widetilde{P}
      &=&\phantom{\cup\ }
%      \{                           r \mid r\in P, {\neg a}\neq\head{r}  \}
\{r\in P\mid \head{r}\neq{\neg a}\}
      \\
      &&\cup\
%      \{  \tilde{a}\leftarrow\body{r}\mid r\in P, {\neg a}=   \head{r}  \}
\{\leftarrow \body{r}\cup\{{\neg \widetilde{a}}\}\mid r\in P \text{ and } \head{r}={\neg a}\}
      \\
      &&\cup\
      % \{
      % \leftarrow a,\tilde{a},\quad
      % \tilde{a}\leftarrow{\neg a}
      % \mid
      % {\neg a}\in\head{P}
      %   \}
\{\widetilde{a}\leftarrow{\neg a}\mid r\in P \text{ and } \head{r}={\neg a}\}
    \end{eqnarray*}
    \smallskip
  \item<4->
    A set~$X$ of atoms is a \alert{stable model} of a program~$P$ (with default negation in rule heads) over $\mathcal{A}$,
    \\
    if $X=Y\cap\mathcal{A}$ for some stable model $Y$ of $\widetilde{P}$ over $\mathcal{A}\cup\widetilde{\mathcal{A}}$
  \end{itemize}
\end{frame}
% ----------------------------------------------------------------------
%
%%% Local Variables:
%%% mode: latex
%%% TeX-master: "../../main"
%%% End:
